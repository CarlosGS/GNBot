%\vspace*{-2cm}
\chapter*{Resumen}
\vspace*{-1.5cm}





En este trabajo de investigaci\'{o}n se aborda el dise\~{n}o de una plataforma rob\'{o}tica orientada a la implementaci\'{o}n de estrategias de b\'{u}squeda cooperativa bio-inspiradas. En particular, tanto el proceso de dise\~{n}o de la parte electr\'{o}nica como hardware se han enfocado hacia la validaci\'{o}n en entornos reales de algoritmos capaces de afrontar problemas de b\'{u}squeda con incertidumbre, como lo es la b\'{u}squeda de fuentes de olor que presentan variaci\'{o}n espacial y temporal. Este tipo de problemas pueden ser resueltos de forma m\'{a}s eficiente con el empleo de enjambres con una cantidad razonable de robots, y por tanto la plataforma ha sido desarrollada utilizando componentes de bajo coste.
Esto ha sido posible por la combinaci\'{o}n de elementos estandarizados -como la placa controladora Arduino y otros sensores integrados- con piezas que pueden ser fabricadas mediante una impresora 3D atendiendo a la filosof\'{i}a del hardware libre (open-source).


Entre los requisitos de dise\~{n}o se encuentran adem\'{a}s la eficiencia energ\'{e}tica -para maximizar el tiempo de funcionamiento de los robots-, su capacidad de posicionamiento en el entorno de b\'{u}squeda, y la integraci\'{o}n multisensorial -con la inclusi\'{o}n de una nariz electr\'{o}nica, sensores de luminosidad, distancia, humedad y temperatura, as\'{i} como una br\'{u}jula digital-.
Tambi\'{e}n se aborda el uso de una estrategia de comunicaci\'{o}n adecuada basada en ZigBee.
El sistema desarrollado, denominado GNBot, se ha validado tanto en los aspectos de eficiencia energ\'{e}tica como en sus capacidades combinadas de posicionamiento espacial y de detecci\'{o}n de fuentes de olor basadas en disoluciones de etanol.

La plataforma presentada -formada por el GNBot, su placa electr\'{o}nica GNBoard y la capa de abstracci\'{o}n software realizada en Python- simplificar\'{a} por tanto el proceso de implementaci\'{o}n y evaluaci\'{o}n de diversas estrategias de detecci\'{o}n, b\'{u}squeda y monitorizaci\'{o}n de odorantes, con la estandarizaci\'{o}n de enjambres de robots provistos de narices artificiales y otros sensores multimodales.


\vspace*{-0.2cm}
\section*{Palabras Clave}
\vspace*{-0.2cm}
Plataforma rob\'{o}tica, b\'{u}squeda de olores, robots cooperativos, estrategias de localizaci\'{o}n, nariz electr\'{o}nica, fuentes de olor, rob\'{o}tica de enjambres, bio-inspiraci\'{o}n, vuelos de L\'{e}vy, Python, hardware libre, printbots, impresi\'{o}n 3D, ZigBee, comunicaci\'{o}n a distancia, OpenCV, visi\'{o}n por ordenador, localizaci\'{o}n de robots

%-------------------------------------------------------------
%\begin{center}
%\line(1,0){250}
%\end{center}
%\vspace*{-2cm}
\chapter*{Abstract}
\vspace*{-1cm}










This research work addresses the design of a robotic platform oriented towards the implementation of bio-inspired cooperative search strategies.
In particular, the design processes of both the electronics and hardware have been focused towards the real-world validation of algorithms that are capable of tackling search problems that have uncertainty, such as the search of odor sources that have spatio-temporal variability.
These kind of problems can be solved more efficiently with the use of swarms formed by a considerable amount of robots, and thus the proposed platform makes use of low cost components.
This has been possible with the combination of standardized elements -as the Arduino controller board and other integrated sensors- with custom parts that can be manufactured with a 3D printer attending to the open-source hardware philosophy.

Among the design requirements is the energy efficiency -in order to maximize the working range of the robots-, their positioning capability within the search environment, and multiple sensor integration -with the incorporation of an artificial nose, luminosity, distance, humidity and temperature sensors, as well as an electronic compass-.
Another subject that is tackled is the use of an efficient wireless communication strategy based on ZigBee.
The developed system, named GNBot, has also been validated in the aspects of energy efficiency and for its combined capabilities for autonomous spatial positioning and detection of ethanol-based odor sources.

The presented platform -formed by the GNBot, the GNBoard electronics and the abstraction layer built in Python- will thus simplify the processes of implementation and evaluation of various strategies for the detection, search and monitoring of odorants with conveniently standardized robot swarms provided with artificial noses and other multimodal sensors.

\vspace*{-0.2cm}
\section*{Keywords}
\vspace*{-0.2cm}
Robotic platform, olfactory search, cooperative robots, search strategies, machine olfaction, odor sources, swarm robotics, bio-inspiration, L\'{e}vy walks, Python, open-source hardware, printbots, 3D-printing, ZigBee, remote communications, OpenCV, computer vision, localization of robots

