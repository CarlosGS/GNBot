\chapter{Glossary}

\begin{itemize}

\item{\textbf{Arduino}: A family of open-source boards generally designed around a single 8-bit Atmel AVR microcontroller. Current models feature a USB interface, analog and digital I/O pins which allows the user to attach various extension boards. (\url{http://www.arduino.cc/})}

\item{\textbf{Arduino shield}: Arduino and Arduino-compatible boards make use of shields - printed circuit expansion boards that plug into the normally supplied Arduino pin-headers.}

\item{\textbf{Ethanol} \emph{(a.k.a. ethyl alcohol or pure alcohol)}: A volatile, flammable, colorless liquid with the structural formula $CH_3CH_2OH$, often abbreviated as $C_2H_5OH$ or $C_2H_6O$. One of its occurrences in nature is as a byproduct of the metabolic process of yeast.}

\item{\textbf{GPS} \emph{(Global Positioning System)}: Space-based satellite navigation system that provides location and time information in all weather conditions, anywhere on or near the Earth where there is an unobstructed line of sight to four or more GPS satellites.}

\item{\textbf{L\'{e}vy walk} \emph{(a.k.a. L\'{e}vy flight)}: A random walk, or path that consists of a succession of random steps, in which the step-lengths have a probability distribution that is heavy-tailed. This means that L\'{e}vy walks combine short steps with longer trajectories, a search behavior that has been observed in many species including sharks, honeybees, and even humans.}

\item{\textbf{Odor Sensor} \emph{(a.k.a. artificial nose, gas sensor)}: Sensing element prominently featured in gas detection equipment in the fields of safety, health, control systems, and instrumentation. Most common odor sensors are based on tin dioxide semiconductor \cite{MillerBakrania06nanostructured}, and one of the leading manufacturers of this kind of technology is Figaro (\url{http://www.figarosensor.com/})}

\item{\textbf{OpenCV} \emph{(Open Source Computer Vision Library)}: An open-source programming library with more than 2500 optimized algorithms, which include a comprehensive set of both classic and state-of-the-art computer vision and machine learning algorithms. (\url{http://opencv.org/})}

\item{\textbf{PCB} \emph{(Printed Circuit Board)}: Structure for mechanically supporting and electrically connecting electronic components with conductive tracks, pads and other features etched from copper sheets laminated onto a non-conductive substrate. PCBs can be single sided (one copper layer), double sided (two copper layers) or multi-layer.}

\item{\textbf{Plume} \emph{(a.k.a. gas plume, odor plume, filamental concentration)}: In hydrodynamics, a plume is a mass of fluid moving through another, and which can be distinguished from the surrounding matter for its different temperature or composition. Plumes evolve in time according to its momentum (inertia), buoyancy (density differences), and diffussion properties.}

\item{\textbf{PrintBots} \emph{(PRINTable roBOTS)}: The family of robots that are open-source and can be manufactured using a low-cost 3D printer. PrintBots are oriented to the community: people in different countries can download and print each of the parts that make a PrintBot, and also modify them and re-share the improvements with the rest of the world. (as defined in \cite{GonzalezValero11miniskybot,ValeroGonzalez12creativity,Garcia-Saura2012,ValeroGonzalezOOML12})}

\item{\textbf{Python}: A widely used general-purpose, high-level programming language that is cross-platform. Its design philosophy emphasizes code readability, and its syntax allows programmers to express concepts in fewer lines of code than what would be possible in languages such as C. (\url{https://www.python.org/})}

\item{\textbf{Servomotor}: Rotatory actuator type that consists of a suitable motor coupled to a sensor for position feedback. Continuous-rotation servomotors do not generally have control over the angular position and can regulate the rotation speed instead. Servomotors are widely used in applications such as robotics, CNC machinery or automated manufacturing.}

\item{\textbf{Transient response} \emph{(a.k.a. natural response)}: The response of a system to a change from equilibrium. The transient response is not necessarily tied to ``on/off'' events but to any event that affects the equilibrium of the system. The \emph{impulse response} and \emph{step response} are transient responses to a specific input (an impulse and a step, respectively).}

\item{\textbf{ZigBee} \emph{(IEEE 802.15.4)}: A specification for a suite of high level communication protocols used to create personal area networks built from small, low-power digital radios. Though low-powered, ZigBee devices can transmit data over long distances by passing data through intermediate devices to reach more distant ones, creating a mesh network. (\url{http://www.zigbee.org/})}

\end{itemize}

\newpage \thispagestyle{empty} % Página vacía

